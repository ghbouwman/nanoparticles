\documentclass{article}
\usepackage{graphicx} % Required for inserting images

\usepackage{amsmath}
\usepackage{amssymb}

\usepackage{circuitikz}


\usepackage{geometry}
 \geometry{
 a4paper,
 total={170mm,257mm},
 left=0mm,
 right=0mm,
 top=0mm,
 bottom=0mm,
 }


\DeclareMathOperator\erf{erf}

\title{Nanoscience Project}
\author{Hans Bouwman}

\begin{document}

\maketitle

\section{Theory}

\begin{circuitikz} \draw
(0,0) to[battery, l_=$V$] (8,0)
    to[R, l_=$R_S$] (8,-4) -- (7,-4)
    -- (7,-6) to[R, l_=$R_G/R_F$] (4,-6) to[normal open switch] (1,-6) -- (1,-4)
    (7,-4) to[R, *-*, l_=$R_C$] (1,-4) -- (0,-4) -- (0,0)
;
\end{circuitikz}




\subsection{Gap}
\[R_{total} = R_S+R_C\]
By voltage divider we get the voltage over the gap:
\[V_G=V\frac{R_C}{R_S+R_C}\]
Suppose one of the sides of the gap, the electric force it experiences depends on the voltage and gap size $x$:
\[\vec{\mathbb{E}}=\frac{V_F}{x}\]
since the gap closes from both sides we have a factor of 2 in the acceleration, we find the ODE
\[\Ddot{x}=-\frac{k}{x}\]
where
\[k=\frac{2qV_F}{m}\]
TODO: can we say somthing about $q$ and $m$\\
we solve the equation as an intial value problem with
\[x(0)=x_0\text{ and }\dot{x}(0)=0\]
the ODE itself yields:
\[x(t)=\exp\left\{\frac{c_1-2k\erf^{-1}\left(\pm|c_2+t|\sqrt{\frac{2k}{\pi}e^{-c_1/k}}\right)^2}{2k}\right\}\]
We make the \textit{ansatz} $c_2=0$ and only consider positive quantities.
\[x(t)=\exp\left\{\frac{c_1-2k\erf^{-1}\left(t\sqrt{\frac{2k}{\pi}e^{-c_1/k}}\right)^2}{2k}\right\}\]
\[x(t)=\exp\left\{\frac{c_1}{2k}\right\}\cdot
\exp\left\{{-\erf^{-1}\left(t\sqrt{\frac{2k}{\pi}e^{-c_1/k}}\right)^2}\right\}\]
Since $\erf^{-1}(0)=0$:
\[x(0)=\exp\left\{\frac{c_1}{2k}\right\}=x_0\]
we get
\[x(t)=x_0\exp\left\{{-\erf^{-1}\left(t\sqrt{\frac{2k}{\pi}e^{-c_1/k}}\right)^2}\right\}\]
the value for $c_1$ thus also implies
\[c_1=2k\ln{x_0}\]
which we substitute
\[x(t)=x_0\exp\left\{{-\erf^{-1}\left(t\sqrt{\frac{2k}{\pi}e^{-2k\ln{x_0}/k}}\right)^2}\right\}\]
which simplifies to
\[x(t)=x_0\exp\left\{{-\erf^{-1}\left(\frac{t}{x_0}\sqrt{\frac{2k}{\pi}}\right)^2}\right\}\]
substituting the expression for $k$ gives
\[x(t)=x_0\exp\left\{{-\erf^{-1}\left(\frac{2t}{x_0}\sqrt{\frac{qV_G}{\pi m}}\right)^2}\right\}\]
TODO: verify ansatz\\
We know that:
\[\lim_{x\to\pm1}\erf^{-1}{x}=\pm\infty\]
Hence to find $x(t)=0$ we solve:
\[\frac{2t}{x_0}\sqrt{\frac{qV_G}{\pi m}}=1\]
which yields:
\[t=\frac{x_0}{2}\sqrt{\frac{\pi m}{qV_G}}\]





\subsection{Filament}
\[R_{total}=R_S+\frac{1}{\frac{1}{R_C}+\frac{1}{R_F}}\]
\[V_F=V\frac{\frac{1}{\frac{1}{R_C}+\frac{1}{R_F}}
}{
R_{total}
}\]
\[=\frac{V}{1+R_S\left(\frac{1}{R_C}+\frac{1}{R_F}\right)}\]
The Joule heating over the filament is:
\[P_J=\frac{V_F^2}{R_F}\]

\[=\frac{\frac{V^2}{R_F}}{1+2R_S\left(\frac{1}{R_C}+\frac{1}{R_F}\right)+R_S^2\left(\frac{1}{R_C}+\frac{1}{R_F}\right)^2}\]


% \[=\frac{V^2}{R_F+2R_S\left(\frac{R_F}{R_C}+1\right)
% +R_S^2R_F\left(\frac{1}{R_C}+\frac{1}{R_F}\right)^2}\]

% \[=\frac{V^2}{R_F+2R_S\left(\frac{R_F}{R_C}+1\right)
% +R_S^2\left(\frac{R_F}{R_C^2}+\frac{R_F}{R_C}+\frac{1}{R_F}\right)}\]

if we assume there is not temperature dependent resistance, we find the energy in the filament at time $t$ after the formation of the filament
\[E_J=P_Jt=\frac{V_F^2}{R_F}t\]
so
suppose the filament is destroyed at an energy $E_{\max}$
\[t=\frac{E_{\max}R_F}{V_F^2}\]

\section{Tunnelling}

\[-\frac{\hbar^2}{2m_e} \frac{{d^2}}{{dx^2}}\Psi(x) + V(x)\Psi(x) = E\Psi(x)\]
\[V(0)=W_{\text{Mo}}\]
\[V(w)=eV_{bias}+W_{\text{Mo}}\]
\[\implies V(x)=\left(\frac{eV_{bias}}{w}\right)x+W_{\text{Mo}}=e\mathcal{E}_{\text{bias}}+W_{\text{Mo}}\]

For the barrier itself:
\[-\frac{\hbar^2}{2m_e} \frac{{d^2}}{{dx^2}}\Psi(x) + \left[e\mathcal{E}_{\text{bias}}x+W_{\text{Mo}}\right]\Psi(x) = E\Psi(x)\]

\[\frac{{d^2}}{{dx^2}}\Psi(x) = \left[\left(\frac{2m_eeV_{bias}}{\hbar^2w}\right)x+\frac{2m_e(W_{\text{Mo}}-E)}{\hbar^2}\right]\Psi(x)\]
let $\alpha=\frac{2m_e}{\hbar^2}$
\[\frac{{d^2}}{{dx^2}}\Psi(x) = \left[\alpha e\mathcal{E}_{\text{bias}}x+\alpha(W_{\text{Mo}}-E)\right]\Psi(x)\]

Solution from WolframAlpha:
\[\frac{{d^2y}}{{dx^2}}=(mx+c)y\]
\[m=\alpha e\mathcal{E}_{\text{bias}}\]
\[c=\alpha(W_{\text{Mo}}-E)\]
\[k_1\text{Ai}(z)+k_2\text{Bi}(z)\]
\[z=\frac{mx+c}{m^{2/3}}\]

\[=\left[\left(\alpha e\mathcal{E}_{\text{bias}}\right)x+\alpha(W_{\text{Mo}}-E)\right]
\cdot\left(\alpha e\mathcal{E}_{\text{bias}}\right)^{-2/3}\]

\[=\left(\alpha e\right)^{1/3}\cdot\left[\mathcal{E}_{\text{bias}}x+\frac{W_{\text{Mo}}-E}{e}\right]
\cdot\mathcal{E}_{\text{bias}}^{-2/3}\]

$x=0$ gives:
\[z=\frac{c}{m^{2/3}}\]
\[=\alpha(W_{\text{Mo}}-E)\cdot\left(\alpha e\mathcal{E}_{\text{bias}}\right)^{-2/3}\]

$x=w$ gives:
\[\mathcal{E}\]

okay but actually it's just:




\[
\frac{1}{R_g} = G_g = G_0 \cdot \exp \left(d_g\sqrt{\frac{W_{\text{Mo}} e_m}{2\hbar^2}} \right)
\]

\[
\sqrt{\frac{W_{\text{Mo}} e_m}{2\hbar^2}} = 5.52\text{ nm}^{-1} = \frac{1}{0.18 \text{ nm} } = \frac{1}{l_\tau}
\]

\[
U=IR\implies I_g=G_gU_g
\]

\[
U=V_g
\]

\[
F_\text{res}=qZ^*j\rho=qZ^*\mathbb{E}
\]

\[
\ddot{x}_\text{res}=\frac{qZ^*\mathbb{E}}{m}=\frac{qZ^*V_g}{mx}
\]

\[
\exp\left(\frac{d_0}{l_\tau}\exp\left(-\erf^{-1}\left(
\frac{2t}{d_0}\sqrt{\frac{qV_g}{\pi m}}
\right)^2\right)\right)
% \approx\exp\left(\frac{d_0}{l_\tau}\left(1-x^2\right)\right)
% \approx x^2(1-e^{d_0/l_\tau}) + e^{d_0/l_\tau}
\]

\end{document}

